\chapter{Conclusion}
Rispetto alla soluzione presentata lo scorso anno, il progetto NCIS ha subito significativi miglioramenti che ne aumentano la robustezza e l’aderenza agli scenari reali di sicurezza SDN. In primo luogo, il problema dell’over blocking è stato risolto passando da un blocco indiscriminato delle porte degli switch a un controllo granulare sui singoli flussi e indirizzi MAC, con l’adozione di whitelist e blocklist. Questo approccio riduce drasticamente l’impatto sul traffico legittimo e migliora l’esperienza degli utenti. Inoltre, le soglie di rilevamento statiche sono state sostituite da meccanismi adattivi basati su medie mobili e percentili, che permettono al sistema di reagire dinamicamente alle variazioni del traffico di rete, riducendo i falsi positivi. L’architettura è stata resa modulare, separando il monitoraggio, la decisione e l’enforcement in thread distinti: questa scelta incrementa la manutenibilità e facilita l’estensione futura del sistema. Dal punto di vista della detection, il sistema ora è in grado di riconoscere pattern di attacco più complessi, come quelli bursty o distribuiti, grazie all’introduzione di nuove metriche come la varianza del traffico e l’analisi degli intervalli tra i pacchetti. Infine, la gestione delle policy di blocco e sblocco è stata resa più flessibile tramite l’adozione di strategie di backoff esponenziale e sblocco progressivo, evitando penalizzazioni ingiustificate e riducendo il rischio di rientro prematuro degli attaccanti.\par
In prospettiva futura, il sistema può essere ulteriormente arricchito con algoritmi di machine learning per la detection, integrazione con sistemi di threat intelligence e supporto a protocolli di orchestrazione avanzata. Il lavoro svolto rappresenta una base solida per la ricerca e lo sviluppo di soluzioni di sicurezza SDN sempre più efficaci e adattive.

\section{Sintesi delle Decisioni Progettuali}

Il progetto NCIS è stato sviluppato seguendo un approccio modulare, con la suddivisione delle responsabilità tra i diversi file principali. Il controller (\texttt{controller.py}) gestisce l’orchestrazione dei moduli e gli eventi OpenFlow, avviando thread dedicati per monitoraggio, detection, mitigazione e API REST. Il monitor (\texttt{monitor.py}) raccoglie periodicamente statistiche granulari dagli switch, fornendo dati aggiornati per la detection. Il detector (\texttt{detector.py}) analizza le statistiche tramite plugin, utilizzando soglie adattive per rilevare anomalie e notificando il mitigator. Il mitigator (\texttt{mitigator.py}) applica regole di blocco e sblocco granulari, gestendo lo sblocco progressivo e il logging delle decisioni. L’API (\texttt{api.py}) espone endpoint REST per la gestione dinamica delle regole.

Le scelte architetturali sono state guidate dalla necessità di garantire modularità, estendibilità e reattività. L’uso di thread separati assicura monitoraggio e risposta continua, mentre le soglie adattive e il blocco granulare riducono i falsi positivi e l’impatto sul traffico legittimo. La documentazione delle decisioni e la presenza di logging facilitano audit, debugging e future estensioni.

La soluzione è stata progettata per essere facilmente estendibile: nuovi plugin di detection, criteri di mitigazione e endpoint API possono essere aggiunti senza modificare la struttura esistente. Il sistema è pronto per ulteriori sviluppi e integrazioni, mantenendo la flessibilità necessaria per adattarsi a scenari e minacce emergenti.
