\chapter{Conclusion}
Rispetto alla soluzione presentata lo scorso anno, il progetto NCIS ha subito significativi miglioramenti che ne aumentano la robustezza e l'aderenza agli scenari reali di sicurezza SDN. In primo luogo, il problema dell'over blocking è stato risolto passando da un blocco indiscriminato delle porte degli switch a un controllo granulare sui singoli flussi e indirizzi MAC, con l'adozione di whitelist e blocklist. Questo approccio riduce drasticamente l'impatto sul traffico legittimo e migliora l'esperienza degli utenti. Inoltre, le soglie di rilevamento statiche sono state sostituite da meccanismi adattivi basati su medie mobili e percentili, che permettono al sistema di reagire dinamicamente alle variazioni del traffico di rete, riducendo i falsi positivi. L'architettura è stata resa modulare, separando il monitoraggio, la decisione e l'enforcement in thread distinti: questa scelta incrementa la manutenibilità e facilita l'estensione futura del sistema. Dal punto di vista della detection, il sistema ora è in grado di riconoscere pattern di attacco più complessi, come quelli bursty o distribuiti, grazie all'introduzione di nuove metriche come la varianza del traffico e l'analisi degli intervalli tra i pacchetti. La gestione delle policy di blocco e sblocco è stata resa più flessibile tramite l'adozione di strategie di backoff esponenziale e sblocco progressivo, evitando penalizzazioni ingiustificate e riducendo il rischio di rientro prematuro degli attaccanti.\par
Una delle innovazioni più significative introdotte è il superamento del limite del controller-centric blocking attraverso l'implementazione di decisioni collaborative. Il sistema ora supporta una struttura dati condivisa per le policy di blocco, permettendo a moduli esterni, sistemi di threat intelligence e amministratori di contribuire alle decisioni di sicurezza. Questa architettura collaborativa rappresenta un passo avanti fondamentale verso un ecosistema di sicurezza SDN più integrato e flessibile.\par
In prospettiva futura, il sistema può essere ulteriormente arricchito con algoritmi di machine learning per la detection, integrazione avanzata con sistemi di threat intelligence distribuiti e supporto a protocolli di orchestrazione di sicurezza collaborativa. Il lavoro svolto rappresenta una base solida per la ricerca e lo sviluppo di soluzioni di sicurezza SDN sempre più efficaci e adattive.

\section{Sintesi delle Decisioni Progettuali}

Il progetto NCIS è stato sviluppato seguendo un approccio modulare, con la suddivisione delle responsabilità tra i diversi file principali. Il controller (\texttt{controller.py}) gestisce l'orchestrazione dei moduli e gli eventi OpenFlow, avviando thread dedicati per monitoraggio, detection, mitigazione e API REST, oltre all'integrazione opzionale di moduli esterni di sicurezza. Il monitor (\texttt{monitor.py}) raccoglie periodicamente statistiche granulari dagli switch, fornendo dati aggiornati per la detection. Il detector (\texttt{detector.py}) analizza le statistiche tramite plugin, utilizzando soglie adattive per rilevare anomalie e notificando il mitigator. Il mitigator (\texttt{mitigator.py}) applica regole di blocco e sblocco granulari, gestendo lo sblocco progressivo e coordinando le decisioni collaborative tramite shared blocklist e external policies. L'API (\texttt{api.py}) espone endpoint REST estesi per la gestione dinamica delle regole e delle policy collaborative. Il modulo di sicurezza esterno (\texttt{external\_security\_module.py}) dimostra l'integrazione con sistemi di threat intelligence e policy amministrative.

Le scelte architetturali sono state guidate dalla necessità di garantire modularità, estendibilità e reattività. L'uso di thread separati assicura monitoraggio e risposta continua, mentre le soglie adattive e il blocco granulare riducono i falsi positivi e l'impatto sul traffico legittimo. L'introduzione del supporto per decisioni collaborative rappresenta un avanzamento significativo, permettendo l'integrazione con sistemi esterni di sicurezza e superando il limite del controller-centric blocking. La documentazione delle decisioni e la presenza di logging facilitano audit, debugging e future estensioni.

La soluzione è stata progettata per essere facilmente estendibile: nuovi plugin di detection, criteri di mitigazione, moduli di sicurezza esterni e endpoint API possono essere aggiunti senza modificare la struttura esistente. Il sistema è pronto per ulteriori sviluppi e integrazioni, mantenendo la flessibilità necessaria per adattarsi a scenari e minacce emergenti.

\section{Contributi e Innovazioni}

Il progetto NCIS introduce diversi contributi significativi nel campo della sicurezza SDN:

\begin{enumerate}
    \item \textbf{Architettura Collaborativa}: Prima implementazione di un sistema SDN che supporta decisioni di blocco collaborative, superando il limite tradizionale del controller-centric blocking.
    
    \item \textbf{Mitigazione Granulare Adattiva}: Combinazione di blocco a livello di flusso con soglie adattive, riducendo significativamente l'over-blocking e i falsi positivi.
    
    \item \textbf{Estendibilità Strutturata}: Framework modulare che facilita l'integrazione di nuovi algoritmi di detection, strategie di mitigazione e sistemi esterni.
    
    \item \textbf{Threat Intelligence Integration}: Dimostrazione pratica di come sistemi di threat intelligence possano contribuire dinamicamente alle policy di sicurezza SDN.
    
    \item \textbf{API Collaborative}: Set completo di endpoint REST per la gestione collaborativa delle policy, abilitando l'integrazione con ecosistemi di sicurezza più ampi.
\end{enumerate}

\section{Limitazioni e Lavori Futuri}

Nonostante i significativi miglioramenti, alcune limitazioni rimangono e aprono la strada a futuri sviluppi:

\begin{itemize}
    \item \textbf{Scalabilità Distributiva}: Il sistema attuale funziona con un singolo controller; futuri sviluppi potrebbero estendere l'architettura collaborativa a controller distribuiti.
    
    \item \textbf{Machine Learning Integration}: L'integrazione di algoritmi di apprendimento automatico potrebbe migliorare ulteriormente la detection di pattern complessi.
    
    \item \textbf{Performance Optimization}: Ottimizzazioni delle strutture dati condivise potrebbero ridurre la latenza in scenari ad alto traffico.
    
    \item \textbf{Security Policy Orchestration}: Sviluppo di protocolli standardizzati per l'orchestrazione di policy di sicurezza tra sistemi eterogenei.
\end{itemize}

\section{Conclusioni Finali}

Il progetto NCIS rappresenta un avanzamento significativo nell'evoluzione dei sistemi di sicurezza SDN, introducendo per la prima volta un approccio collaborativo alle decisioni di blocco che supera i limiti delle soluzioni controller-centriche. L'architettura modulare, le soglie adattive e la mitigazione granulare forniscono una base robusta per la protezione di reti SDN in scenari reali.\par
L'implementazione delle decisioni collaborative apre nuove possibilità per l'integrazione di sistemi di sicurezza eterogenei, creando un ecosistema più ricco e flessibile per la protezione delle infrastrutture di rete. La validazione sperimentale conferma l'efficacia dell'approccio e la sua idoneità per deployment in ambienti produttivi.\par
Il lavoro svolto costituisce una base solida per future ricerche e sviluppi nel campo della sicurezza SDN collaborativa, contribuendo all'evoluzione verso sistemi di protezione sempre più intelligenti e adattivi.

\section{Sintesi delle Decisioni Progettuali}

Il progetto NCIS è stato sviluppato seguendo un approccio modulare, con la suddivisione delle responsabilità tra i diversi file principali. Il controller (\texttt{controller.py}) gestisce l’orchestrazione dei moduli e gli eventi OpenFlow, avviando thread dedicati per monitoraggio, detection, mitigazione e API REST. Il monitor (\texttt{monitor.py}) raccoglie periodicamente statistiche granulari dagli switch, fornendo dati aggiornati per la detection. Il detector (\texttt{detector.py}) analizza le statistiche tramite plugin, utilizzando soglie adattive per rilevare anomalie e notificando il mitigator. Il mitigator (\texttt{mitigator.py}) applica regole di blocco e sblocco granulari, gestendo lo sblocco progressivo e il logging delle decisioni. L’API (\texttt{api.py}) espone endpoint REST per la gestione dinamica delle regole.

Le scelte architetturali sono state guidate dalla necessità di garantire modularità, estendibilità e reattività. L’uso di thread separati assicura monitoraggio e risposta continua, mentre le soglie adattive e il blocco granulare riducono i falsi positivi e l’impatto sul traffico legittimo. La documentazione delle decisioni e la presenza di logging facilitano audit, debugging e future estensioni.

La soluzione è stata progettata per essere facilmente estendibile: nuovi plugin di detection, criteri di mitigazione e endpoint API possono essere aggiunti senza modificare la struttura esistente. Il sistema è pronto per ulteriori sviluppi e integrazioni, mantenendo la flessibilità necessaria per adattarsi a scenari e minacce emergenti.
