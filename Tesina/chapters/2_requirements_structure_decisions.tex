\chapter{Requisiti, Struttura e Decisioni Progettuali}

\section{Miglioramenti rispetto alla soluzione precedente}
Il progetto NCIS nasce dalla volontà di superare i limiti riscontrati nella soluzione presentata lo scorso anno. Di seguito si illustrano i principali difetti individuati e le relative correzioni implementate, con una riflessione sul loro impatto.

\textbf{1. Over blocking:} In passato, il sistema bloccava intere porte degli switch, causando la perdita di traffico legittimo e penalizzando utenti non coinvolti negli attacchi. La nuova implementazione introduce un blocco a livello di flusso e indirizzo MAC, supportato da whitelist e blocklist, riducendo drasticamente l’impatto sugli utenti legittimi e rendendo la mitigazione più precisa.

\textbf{2. Static threshold:} Le soglie di rilevamento erano fissate staticamente sulla topologia, rendendo il sistema poco flessibile e incline a errori in presenza di variazioni di traffico. Ora sono state adottate soglie adattive basate su medie mobili e percentili, che si adattano dinamicamente al traffico, migliorando la capacità di rilevare attacchi senza generare falsi positivi.

\textbf{3. Mancanza di modularità:} La precedente soluzione accorpava monitoraggio, decisione ed enforcement in un unico blocco di codice, rendendo difficile la manutenzione e l’estensione. L’architettura attuale è modulare, con thread separati per monitoring, policy computation ed enforcement, favorendo la chiarezza, la manutenibilità e la scalabilità del sistema.

\textbf{4. Detection limitata a DoS classici:} Il sistema rilevava solo attacchi UDP a bitrate costante, risultando inefficace contro pattern più sofisticati. Ora sono gestiti anche attacchi bursty e distribuiti, grazie all’introduzione di metriche aggiuntive come la varianza del traffico e l’analisi degli intervalli tra i pacchetti, rendendo la detection più robusta e realistica.

\textbf{5. Policy di blocco/sblocco inflessibile:} I tempi di blocco erano troppo rigidi, con rischio di sblocco prematuro o penalizzazione eccessiva. L’adozione di strategie di backoff esponenziale e sblocco progressivo consente una gestione più flessibile e intelligente delle policy, riducendo sia i falsi positivi sia il rischio di rientro degli attaccanti.

Questi miglioramenti sono stati introdotti per rendere il sistema più aderente alle esigenze di sicurezza di una rete SDN reale, aumentando la precisione, la resilienza e la capacità di adattamento del progetto.

\section{Requisiti}
Il sistema è stato progettato per soddisfare i seguenti requisiti fondamentali:
\begin{itemize}
    \item \textbf{Rilevamento in tempo reale}: capacità di individuare attacchi di rete con latenza minima, garantendo una risposta tempestiva.
    \item \textbf{Mitigazione automatica}: applicazione di contromisure efficaci per limitare l’impatto degli attacchi, con particolare attenzione alla minimizzazione dei falsi positivi.
    \item \textbf{Monitoraggio e logging}: raccolta continua di statistiche e eventi di rete, con registrazione dettagliata delle azioni intraprese per facilitare analisi forensi e tuning.
    \item \textbf{Interfaccia REST}: esposizione di endpoint per l’interazione con il sistema, favorendo l’integrazione con dashboard, strumenti di orchestrazione e automazione.
    \item \textbf{Modularità ed estendibilità}: possibilità di aggiungere facilmente nuovi moduli o algoritmi, adattando il sistema a scenari e minacce emergenti.
\end{itemize}
Questi requisiti sono stati definiti sulla base delle criticità riscontrate nelle soluzioni precedenti e delle best practice per la sicurezza in ambienti SDN.\par
In prospettiva, il sistema potrà essere esteso per includere funzionalità di analisi predittiva, integrazione con sistemi di threat intelligence e supporto a protocolli di orchestrazione avanzata.

\section{Design e Struttura del Progetto}
L’architettura del sistema è fortemente modulare: ogni componente è responsabile di un compito specifico e comunica con gli altri tramite chiamate di funzione e endpoint REST. Questa scelta consente di isolare le logiche di detection, mitigazione e monitoraggio, facilitando la manutenzione e l’aggiornamento del sistema. La modularità è stata ulteriormente rafforzata dall’adozione di thread separati per le principali attività, riducendo i colli di bottiglia e migliorando la scalabilità.\par
La progettazione ha tenuto conto anche della necessità di adattarsi a topologie di rete diverse e a pattern di traffico variabili, prevedendo la possibilità di configurare facilmente le metriche di detection e le policy di risposta. L’approccio seguito permette di validare il sistema sia in ambienti di test controllati sia in scenari più complessi e realistici.\par
Questa flessibilità architetturale rappresenta un punto di forza per l’evoluzione futura del progetto.

La struttura del progetto NCIS riflette una precisa scelta architetturale volta a garantire modularità, manutenibilità e chiarezza. Il codice è suddiviso in moduli distinti, ciascuno responsabile di un aspetto fondamentale del sistema: l’API REST (\texttt{api.py}) consente l’interazione esterna e l’integrazione con altri strumenti; il controller (\texttt{controller.py}) coordina le attività di detection e mitigazione; il modulo di rilevamento (\texttt{detector.py}) implementa le logiche adattive per l’identificazione degli attacchi; il mitigatore (\texttt{mitigator.py}) gestisce le strategie di risposta e blocco; il monitor (\texttt{monitor.py}) raccoglie e aggiorna le statistiche di rete necessarie per la detection; infine, i file di topologia (\texttt{topology.py}, \texttt{topology\_new.py}) permettono di simulare diversi scenari di rete per la validazione e il testing.\par
In aggiunta, la presenza di file di documentazione e test (\texttt{Proceedings.md}, \texttt{Tests.md}) garantisce la tracciabilità delle decisioni progettuali e la riproducibilità dei risultati, elementi fondamentali in un contesto accademico e professionale. La struttura modulare consente inoltre di integrare facilmente nuovi algoritmi di detection o strategie di mitigazione, favorendo la sperimentazione e l’aggiornamento continuo.\par
Questa organizzazione è stata pensata per facilitare la collaborazione tra sviluppatori, la revisione da parte di terzi e l’eventuale trasferimento tecnologico verso ambienti produttivi.

\section{Decisioni Progettuali}
Durante lo sviluppo del progetto sono state prese numerose decisioni architetturali e implementative, documentate nei file di proceedings. Questi documenti tracciano il percorso progettuale, le motivazioni delle scelte e le eventuali alternative considerate, fornendo trasparenza e facilitando la collaborazione tra diversi membri del team o la revisione da parte di terzi.\par
La presenza di una documentazione strutturata delle decisioni è un elemento distintivo del progetto, che ne aumenta la qualità e la professionalità.

Il progetto NCIS è stato sviluppato seguendo un approccio modulare, con la suddivisione delle responsabilità tra i diversi file principali. Il controller gestisce l’orchestrazione dei moduli e gli eventi OpenFlow, avviando thread dedicati per monitoraggio, detection, mitigazione e API REST. Il monitor raccoglie periodicamente statistiche granulari dagli switch, fornendo dati aggiornati per la detection. Il detector analizza le statistiche tramite plugin, utilizzando soglie adattive per rilevare anomalie e notificando il mitigator. Il mitigator applica regole di blocco e sblocco granulari, gestendo lo sblocco progressivo e il logging delle decisioni. L’API espone endpoint REST per la gestione dinamica delle regole.\par
Le scelte architetturali sono state guidate dalla necessità di garantire modularità, estendibilità e reattività. L’uso di thread separati assicura monitoraggio e risposta continua, mentre le soglie adattive e il blocco granulare riducono i falsi positivi e l’impatto sul traffico legittimo. La documentazione delle decisioni e la presenza di logging facilitano audit, debugging e future estensioni.\par
La soluzione è stata progettata per essere facilmente estendibile: nuovi plugin di detection, criteri di mitigazione e endpoint API possono essere aggiunti senza modificare la struttura esistente. Il sistema è pronto per ulteriori sviluppi e integrazioni, mantenendo la flessibilità necessaria per adattarsi a scenari e minacce emergenti.
